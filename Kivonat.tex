{\ }

Magyarország, a jelenlegi kutatások szerint, közepes szeizmicitású terület. Korábban kevésbé tartották fontosnak  hazánkban a földrengésre való tervezést, azonban a szeizmikus hatások vizsgálata az Eurocode-8 bevezetésével előtérbe került. A tervezési  igények növekedésével a témára irányuló kutatások  megsokszorozódtak, mivel az új épületek, illetve a már meglévő szerkezetek földrengés elleni védelme új szerkezeti megoldásokat igényel.  

Diplomamunkámban az építőmérnöki szerkezetek  szeizmológiai védelmére irányuló kutatások lehetőségeit mutatom be. Ismertetem a dinamikus számításokhoz alkalmazható numerikus számítási módszereket, valamint egy viszonylag új numerikus számításokkal összekötött kísérleti eljárást,  a hibrid szimulációs módszert.

A szerkezeteket jellemző dinamikus mozgásegyenletek megoldása  a gyakorlatban alkalmazott nagyobb rendszerek esetében analitikusan általában nem lehetséges. A számítógépes kapacitás növekedésével előtérbe kerültek a numerikus időlépéses integráló módszerek, melyekre számos algoritmust fejlesztettek. Dolgozatomban ismertetem elméleti hátterüket, főbb jellemzőiket, és bemutatom a legfontosabb eljárásokat.

Az algoritmusok alapján saját lineáris és nemlineáris megoldó programokat fejlesztek. Bemutatom a lineáris megoldó program használatát gerjesztett rezgések vizsgálatára, valamint szabadrezgés esetében az egyes algoritmusok stabilitásának és pontosságának elemzésére. A nemlineáris megoldó programot alkalmazom statikus probléma vizsgálatára különböző csillapítási tényezők mellet, majd használom dinamikus probléma vizsgálatára is eltérő terhelésekkel, és szemléltetem a nemlinearitás hatását a szerkezet elmozdulásaira. Ezek után elvégzem a programok verifikálását szakirodalmi példákkal.

Dolgozatom második részében  a hibrid szimulációs  eljárást mutatom be. Egy összefoglaló tanulmányban jellemzem az eljárás előnyeit, kivitelezésének lehetőségeit, és a megvalósításának módját. Bemutatom, melyek a szimuláció legfontosabb elemei, és milyen műszerek szükségesek a vizsgálatok elvégzéséhez. 

Egy kiválasztott időlépéses algoritmus alkalmazásával saját hibrid szimulációs programot fejlesztek. A program használatát  bemutatom  egy nemlineáris elemet is tartalmazó szerkezet vizsgálatán keresztül. A feladatban  egy ismert földrengésterhet, az El Centro-t működtetem a szerkezetre, és  leírom a szükséges adatok beolvasásának módját, továbbá elvégzem a hibrid program verifikálását ugyanannak a szerkezetnek a nemlineáris megoldó programban való  számításának eredményeivel. 

Végül bemutatom a hibrid szimulációs eljárás alkalmazását egy képlékeny anyagú szeizmikus szigeteléssel ellátott épület viselkedésének vizsgálatára. Az eredményeket összehasonlítom egy fix megtámasztású épület és egy rugalmas anyagú szeizmikus szigeteléssel ellátott szerkezet  lineáris vizsgálatával kapott eredményekkel. Elvégzem az  eredmények kiértékelését, és ez alapján értékelem a szeizmikus szigetelés szerkezeti viselkedését.



