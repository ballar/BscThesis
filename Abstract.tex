

{\ }

Current research indicates that Hungary is a moderate seismicity area. Earlier in our country earthquake engineering was less important, however, the examination of the seismic effects came to the fore by introducing Eurocode-8. Beside increasing necessity of earthquake engineering, researches relating to the topic have multiplied since  the seismic protection of the new buildings, and existing ones require new structural solutions.

The objective of my thesis is to demonstrate the  opportunities of researching seismic protection of civil engineering structures. Numeric methods used for dynamic calculations is reviewed, and a relatively new method called hybrid simulation  based on connecting numerical calculations with experimental testing is specified as well. 
It is  not generally possible to solve dynamic equation of motion  analytically for larger structures used in practice. Numeric time-stepping methods came to the forefront because of increasing computer capacity. Different integration  algorithms have been developed for structural dynamics. My thesis describes the theoretical background and  the main characteristics of time-stepping process and presents the most relevant integration algorithms.

I developed programs for solving linear and nonlinear problems based on the integration algorithms. I demonstrate how to use the linear solver program to excited vibration tests and to free vibration tests  to examine  the stability and the accuracy of algorithms. The nonlinear solver is used to  test static problem for different damping rates, and it is used to examine a dynamic problem with different loads to show the effect of nonlinearity in displacement  responses. The program is verified by scientific literature examples.

In the second part of my thesis, the hybrid simulation method is presented. The advantages of the process, the options of implementations and the manner of implementation  are summarized in a study. Key components and required instruments  of  simulation  are also introduced.
 
I developed a hybrid simulation program using a selected time-stepping algorithm.  I display  how to use the program through an examination of a  structure containing nonlinear elements. A well-known earthquake load,  namely El Centro, is actuated on the structure and the method of retrieving  necessary data is described. In addition, verification of hybrid program is performed by analysing the same structure with the nonlinear solver program.

Finally  I present the using of  a hybrid simulation method by testing the response of a structure with  base isolation made of ductile material. The results are compared with the results of linear analysis of a fixed-base building and  an isolated building, where base isolation is made of flexible material. Evaluation of the results is performed, and the behaviour of the base isolation is rated according to the results.


