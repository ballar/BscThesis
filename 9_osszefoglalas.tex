\chapter{Összefoglalás}

{\ }

Dolgozatomban egy összefoglaló tanulmányban bemutattam a mozgásegyenletek megoldására alkalmazható időlépéses integrálás elméleti hátterét. Ismertettem a módszerek általános jellemzőit, majd a legfontosabb lineáris és nemlineáris integráló formulákat is. Összefoglaltam, hogy a különböző numerikus számításokat végző programok ezek közül mely integráló formulákat használják.

Az integráló algoritmusok  segítségével saját programokat fejlesztettem a lineáris és a nemlineáris dinamikus  feladatok megoldására, majd prezentáltam használatukat. A lineáris program használatát először egy gerjesztett rezgéses feladaton mutattam be, ahol a gerjesztés és a csillapítás hatását vizsgáltam, majd egy szabadrezgéses feladaton keresztül bemutattam, hogy a program hogyan használható az időlépéses módszerek stabilitásának és pontosságának vizsgálatára. A nemlineáris programnál először azt mutattam meg, hogy a program hogyan használható statikus  probléma vizsgálatára.  A feladatban az adott szerkezetet különböző csillapítási   tényezők  mellett alkalmaztam. Ezek után ismertettem a program használatát dinamikus probléma megoldására is. Szemléltettem a nemlinearitás hatását, valamint vizsgáltam a teherfüggvény módosításának hatását a rendszer elmozdulásaira.

Elvégeztem a lineáris és a nemlineáris programok verifikálását   Chopra  könyvében \cite{chopra} ismertetett szakirodalmi példákkal. A verifikáció mindkét esetben sikeres volt. A lineáris programban ugyanazok az eredmények születtek, mint a könyvben. A nemlineáris program esetében az eredmények között eltérések adódtak, de ezek egyértelműen betudhatók a feladatban alkalmazott anyagmodell hiányos ismertetésének.

Ezek után egy összefoglaló tanulmányban bemutattam a szerkezetek két különálló részre osztott, de összekapcsolt vizsgálatán alapuló  hibrid szimulációs eljárást. Összehasonlítottam más kísérleti eljárásokkal, bemutattam a módszer előnyeit, és a hozzá kapcsolódó főbb kihívásokat. Részleteztem az eljárás  lépéseit, és a szimuláció kivitelezésének különböző módjait.

A Chen-Ricles algoritmus alapján  hibrid programot fejlesztettem. Bemutattam a program használatát egy kétszintes,  nemlineáris merevítéssel ellátott szerkezeten. A szerkezeten az El Centro földrengésterhet működtettem. Elvégeztem a vizsgálatot, majd verifikáltam az eredményeket ugyanennek a szerkezetnek a nemlineáris megoldó programban végzett számításai alapján.

Végül alkalmaztam a hibrid szimulációs eljárást a szeizmikus szigetelések hatásának bemutatására. Az épület viselkedését összehasonlítottam egy a lineáris megoldó programban számított fix megtámasztású szerkezet, valamint egy a szintén a lineáris megoldóval számított szeizmikus szigeteléssel ellátott szerkezet viselkedésével. A hibrid programban vizsgált szerkezetre egy képlékeny anyagmodellt alkalmaztam. Láthattuk, hogy a szeizmikus szigetelés hatása jelentősen csökkenti az épületekben keletkező igénybevételeket.

Diplomamunkámban, a kiírt célkitűzéseknek megfelelően, vizsgálataimban bemutattam a mozgásegyenletek numerikus számításainak lehetőségeit, továbbá ismertettem a hibrid szimuláció elméleti hátterét, részleteztem az előnyeit, programommal pedig demonstráltam az alkalmazhatóságát. A földrengésre való tervezés előtérbe kerülése miatt szükség van új szerkezetek fejlesztésére, és azok viselkedésének vizsgálatára. Erre jelenleg a hibrid szimulációs eljárás a legalkalmasabb. A fentiek alapján elmondhatjuk, hogy az eljárást érdemes  hazánkban is megvalósítani, mert nagy előrelépést jelenthet a szeizmológiai védekezés  magyarországi kutatásaiban. 